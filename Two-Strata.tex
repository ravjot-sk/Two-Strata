\documentclass[a4paper,12pt]{article}
\usepackage{parskip}
\usepackage[utf8]{inputenc}
\usepackage{mathtools,amssymb,amsthm,thmtools,xparse}
\usepackage{enumitem}
\usepackage{hyperref,cleveref}
\usepackage{lipsum}
\usepackage[dvipsnames]{xcolor} 
\usepackage{tikz,tikz-cd}
\usepackage{bm}

\declaretheorem[numberwithin = section]{theorem}
\declaretheorem[sibling = theorem, style = definition]{definition}
\declaretheorem[sibling = theorem]{lemma}
\declaretheorem[sibling = theorem]{proposition}
\declaretheorem[sibling = theorem]{remark}
\newtheorem{ccclaim}{Claim}[subsection]


\newlist{Claim}{description}{2}
\setlist[Claim]{labelindent=2em,leftmargin=*}
\newif\ifInsideClaim
\newcounter{claim}[theorem]
\newcounter{cclaim}[claim]
\renewcommand\theclaim{\arabic{claim}}
\renewcommand\thecclaim{\arabic{claim}.\arabic{cclaim}}
\let\originalqedsymbol\qedsymbol
\newenvironment{claim}{%
	% disable qed symbol if there is no star
	\let\qedsymbol\relax%
	\ifInsideClaim% we have a nested environment
	\refstepcounter{cclaim}%
	\let\theclaimcounter\thecclaim%
	\else%
	\refstepcounter{claim}%
	\let\theclaimcounter\theclaim%
	\InsideClaimtrue%
	\fi%
	\Claim\item[\textbf{Claim \theclaimcounter:}]%
}{\endClaim\InsideClaimfalse\let\qedsymbol\originalqedsymbol}
% \newtheorem{theorem}{Theorem}[section]
% \newtheorem{corollary}[theorem]{Corollary}
% \newtheorem{lemma}[theorem]{Lemma}
% \newtheorem{proposition}[theorem]{Proposition}

% \theoremstyle{definition}
% \newtheorem{example}[theorem]{Example}

% \theoremstyle{definition}
% \newtheorem{definition}[theorem]{Definition}

% \theoremstyle{remark}
% \newtheorem*{remark}{Remark}

\newcommand{\cpx}[1]{#1^\mathbb{C}}
\newcommand{\C}{\mathbb{C}}
\newcommand{\R}{\mathbb{R}}
\newcommand{\tgt}[2]{T_{#1} #2}
\newcommand{\inv}[1]{#1^{-1}}
\newcommand{\closure}[1]{\overline{#1}}
\newcommand{\sslash}{\mathbin{/\mkern-6mu/}}
\newcommand{\del}{\partial}
\newcommand{\delbar}{\bar{\partial}}
\newcommand{\red}[1]{\textcolor{red}{#1}}
\newcommand{\abs}[1]{\lvert #1 \rvert}
\newcommand{\norm}[1]{\lVert #1 \rVert}

\title{Two Strata Case}
\author{Ravjot Singh Kohli}
\date{Apr 2023}

\begin{document}
	\maketitle
	
	\section{K\"ahler Reduction on $\C^n$}
	The simplest example of K\"ahler reduction with an isolated singular point is a linear action. Let $(z^1,\dots,z^n)$ be holomorphic coordinates on $\C^n$. The standard K\"ahler form on $\C^n$ is given by,
	\begin{equation}\label{standardkahler}
		(\omega_{std})_{z} \,=\,  \sum_{k=1}^n \, \frac{i}{2} \, dz^k \wedge d\bar{z}^k
	\end{equation}
	The complex structure in these coordinates is given by 
	\begin{equation}
		J_{std}\, \left(\frac{\partial}{\partial z^j} \right) \,=\, i \, \frac{\partial}{\partial z^j}, \qquad J_{std}\, \left( \frac{\partial}{\partial \bar{z}^j} \right) \,=\, -i \, \frac{\partial}{\partial \bar{z}^j}
	\end{equation}
	The standard Riemannian metric on $\C^n$ and the Hermitian metric, respectively, are given by
	\begin{equation}
		\begin{split}
			g_{std}(z) \,&=\, \omega_{std}(-,J_{std}-) \,=\, \sum_{k=1}^n \, dz^k \odot d\bar{z}^k \\
			h_{std}(z) \,&=\, \sum_{k=1}^n \, dz^k \otimes d\bar{z}^k
		\end{split}
	\end{equation}
	Note that this hermitian structure corresponds to the standard hermitian inner product on the vector space $\C^n$.
	
	Recall that $\C^n\backslash\{0\} \simeq (0,\infty) \times S^{2n-1}$ with coordinates on the right given by polar coordinates $(r,\theta)$. Here $r = \abs{z}$ and $\theta$ denotes the coordinates on $S^{2n-1}$.
	The standard metric in these coordinates takes the form
	\begin{equation}
		g_{std}(r,\theta) \,=\, dr^2 \,+\, r^2 g_{S^{2n-1}}(\theta)
	\end{equation}
	
	Let $G\subset U(n)$ be a compact Lie group acting on $\C^n$ via unitary matrices. Since the action of $U(n)$ is Hamiltonian, the moment map is given by
	\begin{align*}\label{standardmoment}
		\Phi_{std}(z) (A) \,=\, (\omega_{std})_z(A_{\C^n}(z), z)
	\end{align*}
	where $A\in \mathfrak{g}\subset \mathfrak{u}(n)$ is a skew-Hermitian matrix and $A_{\C^n}$ is the vector field generated by the group action.	
	
	We assume that apart from the fixed point set, the $G$-action is free. Hence we have two strata given by the orbit types, $(\C^n)_G$ of orbit type $(G)$ and $(\C^n)_e=(\C^n)\backslash (\C^n)_{G}$ of orbit type $(e)$ where $e\in G$ is the identity element. 
	
	Note that $(\C^n)_G$ is a linear \red{symplectic subspace}. Let $W:= ((\C^n)_G)^\perp$ denote the perpendicular subspace with respect to the standard \red{hermitian} inner product on $\C^n$. Then $W$ is a \red{symplectic subspace} as well. \red{The following is a symplectic, orthogonal, and $G$-invariant decomposition of $\C^n$}
		\begin{lemma}
		The subspaces $(\C^n)_G$ and $W$ are symplectic and complex subspaces of $\C^n$ \red{find simple argument}
	\end{lemma}

	\begin{equation}
		\C^n \,=\, (\C^n)_G \, \oplus \, W
	\end{equation}
	The moment map also decomposes as
	\begin{equation}
		 \Phi_{std} = \Phi_{(\C^n)_G} \,+\, \Phi_W	
	\end{equation}
	where the maps on the right are the restriction of $\Phi_{std}$ to the respective subspaces.

%	\begin{proof}
%		$(\C^n)_G$ is a complex subspace is clear from the fact that if $v_1, v_2 \in (\C^n)_G$ then $a v_1 \,+\, b v_2 \in (\C^n)_G$ for all $a,b\in \C$ since $G$ action is via unitary matrices which are complex linear. $W$ being the orthogonal complement with respect to the \red{hermitian} inner product means that $W$ is again a complex subspace.
%		
%		Assume that the restriction of the symplectic form $\omega_{std}$ to $(\C^n)_G$ is degenerate. Then there exists $p\in (\C^n)_G$ and $v \in \tgt{p}{((\C^n)_G)}$ such that for all $w \in \tgt{p}{((\C^n)_G)}$, 
%		\begin{equation*}
%			\begin{split}
%				0 \,&=\, (\omega_{std})_p(v,w)\\
%				&=\, (g_{std})_p(J_{std}\cdot v, w)\\
%				&=\, (g_{std})_p(v_1, w)
%			\end{split}
%		\end{equation*}
%		 where $v_1 \in \tgt{p}{((\C^n)_G)}$ since $(\C^n)_G$ is a complex subspace. But the Riemannian metric is positive definite on any subspace and so we have a contradiction. Hence $(\C^n)_G$ is a symplectic subspace.
%	\end{proof}
	We can write the stratum $(\C^n)_e$ as the product
	\begin{equation}
		(\C^n)_e \,=\, (\C^n)_G \times (W \backslash\{0\})
	\end{equation}
	For a point $z=(u,w)\in (\C^n)_e$, the Riemannian metric is given
	\begin{equation}
		g_{std} (u,w) \,=\, \sum_{k=1}^m \, du^k \odot d\bar{u}^k \,+\, \sum_{k=1}^l \, dw^k \odot d\bar{w}^k
	\end{equation} 
	
	
	Let $A\in \mathfrak{g} \subset \mathfrak{u}(n)$. The vector field generated on $\C^n$ by the group action, denoted $A_{\C^n}$, is given by
	\begin{equation}\label{vecfield}
		A_{\C^n}(z) \,=\, \frac{d}{dt}\Bigr|_{t=0} \text{exp}(tA)\cdot z \,=\, \begin{cases}
			0 & \text{$z \in (\C^n)_G$} \\
			A \cdot z & \text{$z \in (\C^n)_{e}$}
		\end{cases}
\end{equation}
	
	
	
	\begin{lemma}\label{conelemma}
		The zero level set $Z_{std}:=\inv{\Phi_{std}}(0)$ is a cone,  \textit{i.e.,} $Z_{std} \simeq [0,\infty) \times L$ where $L=Z_{std}\cap S^{2n-1}$
	\end{lemma} 
	\begin{proof}
		Let $p\in Z_{std}$. Consider the scalar multiplication of $\R^+$ on $\C^n$ denoted by $t\cdot p$. Then we have
		\begin{equation*}
			\begin{split}
				\Phi_{std}(t\cdot p) (A) \,&=\, (\omega_{std})_{t\cdot p}(A_{\C^n}(t\cdot p), t\cdot p) \\
				&=\,  (\omega_{std})_p(t \cdot A_{\C^n}(p), t\cdot p)\\
				&=\, t^2 \, \Phi_{std}(p)(A)\\
				&=\, 0
			\end{split}
		\end{equation*}
		where we have used that $(\omega_{std})_{p}$ is independent of the point $p\in \C^n$ and that scalar multiplication on $\C^n$ is linear and commutes with the action of $\mathfrak{u}(n)$.
	\end{proof}
	
	Note that $0\in Z_{std} \subset \C^n$ is the only singular point and $Z_{std}\backslash\{0\}$ is a smooth manifold. The reduced space defined as the quotient $\pi:Z_{std} \to Z_{std}/G=:(\C^n)_0$ has two strata, the point $\pi(0)$ and the rest. Hence outside the point $\pi(0)$, we can talk about the symplectic form $(\omega_{std})_0$ and the Riemannian metric $(g_{std})_0$ on the manifold $(\C^n)_0\backslash\{\pi(0)\}$ coming from smooth K\"ahler reduction. 
	
	Note that the $G$-action on $Z_{std}\backslash\{0\} \simeq (0,\infty) \times L$ acts only on the link $L$ and so 
	\begin{equation}
		(\C^n)_0\backslash\{\pi(0)\} \,\simeq\, (0,\infty) \times (L/G)
	\end{equation}
	Now combining \Cref{quotientmetric}, \Cref{kahlerreduction}, and the above decomposition, we get that the metric on the reduced space can be written as
	\begin{equation}\label{standardquotientmetric}
		(g_{std})_0(r,\phi) \,=\, dr^2 \,+\, r^2 g_{L/G}(\phi)
	\end{equation}
	where $g_{L/G}$ is the quotient metric on manifold $L/G$ (with coordinates $(\phi)$).
	
	\section{Ideal Metric}
	Let $(M,\omega)$ be a symplectic manifold with a Hamiltonian action by a compact Lie group $G$ with moment map $\Phi_M$. Let $H:=G_x$ for a point $x\in M$. The symplectic slice to the point $x$ is defined as 
	\begin{equation}
		V:= (\tgt{x}{(G\cdot x)})^\omega / (\tgt{x}{(G\cdot x)})
	\end{equation}
	This is a symplectic subspace of $\tgt{x}{M}$. Let $\mathfrak{g},\mathfrak{h}$ denote the Lie algebras of groups $G$ and $H$, respectively. We denote the quotient vector space $\mathfrak{m}:= \mathfrak{g}/\mathfrak{h}$.
	
	\begin{theorem}[Prop 2.5, SL]
		A neighbourhood of the orbit $G\cdot x$ in $M$ is $G$-equivariantly symplectormorphic to a neighbourhood of the zero section of the associated bundle $G \times_H (\mathfrak{m}^* \times V)$ with the moment map given by 
		\begin{equation}
			\begin{split}
				\Phi : G \times_H (\mathfrak{m}^* \times V) &\rightarrow \mathfrak{g}^* \\
				[(g,\mu,v)] & \mapsto {Ad}^*(g) (\mu + \Phi_V(v))
			\end{split}
		\end{equation}
	where $\Phi_V$ is the moment map on the symplectic vector space $V$. 
	\end{theorem}  
	The zero level set is given by
	\begin{equation}
		Z:= \inv{\Phi}(0) = G \times_H(\{0\}\times \inv{\Phi_V}(0))
	\end{equation}
	\red{The vector space $V$ is a $H$-representation space }
\end{document}
