\documentclass[a4paper,12pt]{article}
\usepackage{parskip}
\usepackage[utf8]{inputenc}
\usepackage{mathtools,amssymb,amsthm,thmtools,xparse}
\usepackage{enumitem}
\usepackage{hyperref,cleveref}
\usepackage{lipsum}
\usepackage[dvipsnames]{xcolor} 
\usepackage{tikz,tikz-cd}
\usepackage{bm}

\declaretheorem[numberwithin = section]{theorem}
\declaretheorem[sibling = theorem, style = definition]{definition}
\declaretheorem[sibling = theorem]{lemma}
\declaretheorem[sibling = theorem]{proposition}
\declaretheorem[sibling = theorem]{remark}
\newtheorem{ccclaim}{Claim}[subsection]


\newlist{Claim}{description}{2}
\setlist[Claim]{labelindent=2em,leftmargin=*}
\newif\ifInsideClaim
\newcounter{claim}[theorem]
\newcounter{cclaim}[claim]
\renewcommand\theclaim{\arabic{claim}}
\renewcommand\thecclaim{\arabic{claim}.\arabic{cclaim}}
\let\originalqedsymbol\qedsymbol
\newenvironment{claim}{%
	% disable qed symbol if there is no star
	\let\qedsymbol\relax%
	\ifInsideClaim% we have a nested environment
	\refstepcounter{cclaim}%
	\let\theclaimcounter\thecclaim%
	\else%
	\refstepcounter{claim}%
	\let\theclaimcounter\theclaim%
	\InsideClaimtrue%
	\fi%
	\Claim\item[\textbf{Claim \theclaimcounter:}]%
}{\endClaim\InsideClaimfalse\let\qedsymbol\originalqedsymbol}
% \newtheorem{theorem}{Theorem}[section]
% \newtheorem{corollary}[theorem]{Corollary}
% \newtheorem{lemma}[theorem]{Lemma}
% \newtheorem{proposition}[theorem]{Proposition}

% \theoremstyle{definition}
% \newtheorem{example}[theorem]{Example}

% \theoremstyle{definition}
% \newtheorem{definition}[theorem]{Definition}

% \theoremstyle{remark}
% \newtheorem*{remark}{Remark}

\newcommand{\cpx}[1]{#1^\mathbb{C}}
\newcommand{\C}{\mathbb{C}}
\newcommand{\R}{\mathbb{R}}
\newcommand{\tgt}[2]{T_{#1} #2}
\newcommand{\inv}[1]{#1^{-1}}
\newcommand{\closure}[1]{\overline{#1}}
\newcommand{\sslash}{\mathbin{/\mkern-6mu/}}
\newcommand{\del}{\partial}
\newcommand{\delbar}{\bar{\partial}}
\newcommand{\red}[1]{\textcolor{red}{#1}}
\newcommand{\abs}[1]{\lvert #1 \rvert}
\newcommand{\norm}[1]{\lVert #1 \rVert}

\title{Two Strata Case}
\author{Ravjot Singh Kohli}
\date{Apr 2023}

\begin{document}
	\maketitle
	
	\section{K\"ahler Reduction on $\C^n$}
	The simplest example of K\"ahler reduction with an isolated singular point is a linear action. Let $(z^1,\dots,z^n)$ be holomorphic coordinates on $\C^n$. The standard K\"ahler form on $\C^n$ is given by,
	\begin{equation}\label{standardkahler}
		(\omega_{std})_{z} \,=\,  \sum_{k=1}^n \, \frac{i}{2} \, dz^k \wedge d\bar{z}^k
	\end{equation}
	The complex structure in these coordinates is given by 
	\begin{equation}
		J_{std}\, \left(\frac{\partial}{\partial z^j} \right) \,=\, i \, \frac{\partial}{\partial z^j}, \qquad J_{std}\, \left( \frac{\partial}{\partial \bar{z}^j} \right) \,=\, -i \, \frac{\partial}{\partial \bar{z}^j}
	\end{equation}
	The standard metric on $\C^n$ is given by
	\begin{equation}
		g_{std}(z) \,=\, \omega_{std}(-,J_{std}-) \,=\, \sum_{k=1}^n \, dz^k \otimes d\bar{z}^k
	\end{equation}
	Recall that $\C^n\backslash\{0\} \simeq (0,\infty) \times S^{2n-1}$ with coordinates on the right given by polar coordinates $(r,\theta)$. Here $r = \abs{z}$ and $\theta$ denotes the coordinates on $S^{2n-1}$.
	The standard metric in these coordinates takes the form
	\begin{equation}
		g_{std}(r,\theta) \,=\, dr^2 \,+\, r^2 g_{S^{2n-1}}(\theta)
	\end{equation}
	
	Now let $G\subset U(n)$ be a compact Lie group acting on $\C^n$. We assume that apart from the fixed point set $(\C^n)^G$, the action is free. Hence we have two strata given by the orbit types, $(\C^n)^G$ of orbit type $(G)$ and $(\C^n)\backslash (\C^n)^G$ of orbit type $(e)$  where $e\in G$ is the identity element. 
	
	Let $A\in \mathfrak{g} \subset \mathfrak{u}(n)$. The vector field generated on $\C^n$ by the group action, denoted $A_{\C^n}$, is given by \red{vertical line}
	\begin{equation}
		\begin{split}
			A_{\C^n}(z) \,&=\, \frac{d}{dt}\rvert_{t=0} exp(tA)\cdot z\\
				&=\, A \cdot z
		\end{split}
	\end{equation}
	where we have assumed that $z$ is not a fixed point of $G$-action.
	
	The moment map with respect to this action is given by
	\begin{align*}\label{standardmoment}
		\Phi_{std}(z) (A) \,=\, (\omega_{std})_z(A\cdot z, z)
	\end{align*}
	where $A\in \mathfrak{g}\subset \mathfrak{u}(n)$ is a skew-Hermitian matrix. 	
	
	\begin{lemma}\label{conelemma}
		The zero level set $Z_{std}:=\inv{\Phi_{std}}(0)$ is a cone,  \textit{i.e.,} $Z_{std} \simeq [0,\infty) \times L$ where $L=Z_{std}\cap S^{2n-1}$
	\end{lemma} 
	\begin{proof}
		Let $p\in Z_{std}$. Consider the scalar multiplication of $\R^+$ on $\C^n$ denoted by $t\cdot p$. Then we have
		\begin{equation*}
			\begin{split}
				\Phi_{std}(t\cdot p) (A) \,&=\, (\omega_{std})_{t\cdot p}(A\cdot (t\cdot p), t\cdot p) \\
				&=\, t^2\, (\omega_{std})_p(A\cdot p, p)\\
				&=\, t^2 \, \Phi_{std}(p)(A)\\
				&=\, 0
			\end{split}
		\end{equation*}
		where we used that $(\omega_{std})_{p}$ is independent of the point $p\in \C^n$ and that scalar multiplication on $\C^n$ is linear and commutes with the action of $\mathfrak{u}(n)$.
	\end{proof}
	
	Note that $0\in Z_{std} \subset \C^n$ is the only singular point and $Z_{std}\backslash\{0\}$ is a smooth manifold. The reduced space defined as the quotient $\pi:Z_{std} \to Z_{std}/G=:(\C^n)_0$ has two strata, the point $\pi(0)$ and the rest. Hence outside the point $\pi(0)$, we can talk about the symplectic form $(\omega_{std})_0$ and the Riemannian metric $(g_{std})_0$ on the manifold $(\C^n)_0\backslash\{\pi(0)\}$ coming from smooth K\"ahler reduction. 
	
	Note that the $G$-action on $Z_{std}\backslash\{0\} \simeq (0,\infty) \times L$ acts only on the link $L$ and so 
	\begin{equation}
		(\C^n)_0\backslash\{\pi(0)\} \,\simeq\, (0,\infty) \times (L/G)
	\end{equation}
	Now combining \Cref{quotientmetric}, \Cref{kahlerreduction}, and the above decomposition, we get that the metric on the reduced space can be written as
	\begin{equation}\label{standardquotientmetric}
		(g_{std})_0(r,\phi) \,=\, dr^2 \,+\, r^2 g_{L/G}(\phi)
	\end{equation}
	where $g_{L/G}$ is the quotient metric on manifold $L/G$ (with coordinates $(\phi)$).
\end{document}
